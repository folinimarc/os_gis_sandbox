\documentclass[12pt, letterpaper, twoside]{article}

% -- PREABLE
\usepackage[utf8]{inputenc}
\usepackage{graphicx}
\usepackage[parfill]{parskip}

\graphicspath{ {img/} } % path where grahics are stored

% title page parameters
\title{Sandbox \& Data Story - Toolage to support GIS workshops}
\author{Marc Folini}
\date{May 2022}

% -- BODY
\begin{document}

%-------------
% TITLE, ABSTRACT
%-------------
\begin{titlepage}
  \pagenumbering{Roman}
  \setcounter{page}{1}
  % create title page
  \clearpage\maketitle
  \thispagestyle{empty}
  \vspace{2.5cm}
  % abstract
  \begin{abstract}
  \end{abstract}
\end{titlepage}

%-------------
% TABLE OF CONTENT
%-------------
\newpage
\tableofcontents

%-------------
% INTRODUCTION
%-------------
\newpage
\pagenumbering{arabic}
\setcounter{page}{1}
\section{Introduction}
Introductionary educational GIS workshops face a few particular challenges, of which this thesis will focus on two.
Firstly, it is generally desirable that participants can immediately experience theoretical concepts hands-on for
themselves. The challenge here lies in providing all participants with access to the respective software and data with
minimal setup efforts. Secondly, the GIS landscape today consists of many different components covering the whole data
lifecycle from initial data exploration to visualization and distribution. Together with the myriad of commercial and
open source tools for each component, this might be overwhelming to unfamiliar participants.

This thesis tackles these challenges at two levels:
\begin{enumerate}
  \item \textbf{Infrastructure} In order to provide participants access to a variety of software with minimal setup
  effort the concept of a \emph{sandbox environment} is adopted. The requirements and components to be implemented in a
  first version will be chosen to support the content of this thesis' further education curriculum.
  \item \textbf{Content} In order to help participants to make sense of the sandbox components and their interactions,
  we introduce the concept of a \emph{data journey}. Each data journey has a tangible data usecase goal statement that
  participants can identify with, for example: Let's visualize risk of wildfire and serve it as interactive map on a
  website. Starting from given raw data, participants follow the processing of this dataset through the different
  components and experience how they work together to achieve the goal.
\end{enumerate}


%-------------
% REQUIREMENTS
%-------------
\section{Requirements}


%-------------
% SANDBOX
%-------------
\section{Sandbox}

%-------------
% DATA STORY
%-------------
\section{Data Story}

%-------------
% CONCLUSION
%-------------
\section{Conclusion}


\end{document}