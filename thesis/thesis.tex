\documentclass[11pt, a4paper, oneside, parskip=full-]{scrartcl}

% -- PREABLE

% global settings
\renewcommand\labelitemi{-}

% packages
\usepackage[utf8]{inputenc}
\usepackage{graphicx} % to embed figures
\usepackage{float} % use [H] to place fig exactly where specified in code
\usepackage{tabularx} % table with columns that fill width of page using X
\usepackage{booktabs} % table utilities like \midrule
\usepackage{multirow} % span multiple rows in table
\usepackage{enumitem} % pass key-value config to itemize and enumberable
\usepackage{hyperref} % for links and urls

\graphicspath{ {img/} } % path where grahics are stored

% title page parameters
\title{Open-source GIS sandbox and data stories - A new tool for GIS workshops}
\author{Marc Folini}
\date{May 2022}

% -- BODY
\begin{document}

%-------------
% TITLE, ABSTRACT
%-------------
\begin{titlepage}
  \pagenumbering{Roman}
  \setcounter{page}{1}
  % create title page
  \clearpage\maketitle
  \thispagestyle{empty}
  % abstract
  \begin{abstract}
    It is desirable in GIS workshops to supplement the theoretical concepts with
    hands-on exercises in order for participants to become an active part in the
    learning experience. This requires that participants have access to an
    environment that provides the necessary software and data - ideally with
    minimal setup effort, independent of the system they use and easy to
    uninstall without leaving traces after the workshop. Different approaches
    were evaluated, and container technology was found to be a good fit to
    abstract away the hassle of setting up the tools and infrastructure. A proof
    of concept of what will be referred to as \emph{sandbox environment} was
    implemented and published as open source project.

    Furthermore, this thesis explored the concept of \emph{data stories}. A data
    story provides selected raw data, a motivating tangible goal and
    instructions that connect the raw data to the goal. This allows participants
    to fully focus on how the available tools interact with data and other tools
    to achieve something useful. An example data story was implemented whereby
    open data from the city of Zürich is used to derive a map showing what
    percentage of roads in each district are suitable for biking. Participants
    learn about GDAL's command line tools to inspect datasets and load them into
    PostGIS, where spatial processing and aggregation is performed. The result
    is made available as a new dataset to be consumed and styled in QGIS.
  \end{abstract}
\end{titlepage}

%-------------
% TABLE OF CONTENT
%-------------
\newpage
\tableofcontents

%-------------
% INTRODUCTION
%-------------
\newpage
\pagenumbering{arabic}
\setcounter{page}{1}
\section{Introduction}
Educational GIS\footnote{Geographic Information System} workshops face a few
particular challenges, two of which are central to this thesis. Firstly, it is
generally desirable that participants can immediately experience theoretical
concepts hands-on for themselves. The challenge here lies in providing all
participants with access to the respective software and data with minimal setup
efforts. Secondly, the GIS landscape today consists of many components covering
the whole data lifecycle from initial data exploration to visualization and
distribution. Together with the myriad of commercial and open source tools for
each component, this might be overwhelming to unfamiliar participants.

This thesis was written as part of the CAS RIS 2021/22\footnote{Certificate of
advanced studies (CAS) in GIS (in German RIS) offered by ETH Zurich.}. It
explores the two above-mentioned challenges in the context of the CAS and aims
to address them on two different levels:
\begin{enumerate}
  \item \textbf{Infrastructure} In order to provide participants access to a
  variety of software with minimal setup effort the concept of a \emph{sandbox
  environment} is adopted. The requirements and components to be implemented in
  a first version were chosen to support the content of this thesis' further
  education curriculum.
  \item \textbf{Content} In order to help participants to make sense of the
  sandbox components and their interactions, the concept of a \emph{data story}
  was explored. Each data story has a tangible data use case goal statement that
  participants can identify with, for example: Let's visualize risk of wildfire
  and serve it as interactive map on a website. Starting from given raw data,
  participants follow the processing of this dataset through the different
  components and experience how they work together to achieve the goal.
\end{enumerate}


%-------------
% REQUIREMENTS
%-------------
\section{Requirements} \label{sectionrequirements} The content of the CAS RIS
was analyzed and assessed for potential hands-on exercises. From these findings
functional and non-functional requirements\footnote{Functional requirements
specify the tasks an application must be able to perform. Non-functional
requirements specify additional aspects about the application performing the
task, for example usability, look and feel or performance. } were then derived.

% Content analyis
%-------------
\subsection{Content analysis}
At the time of writing the course consisted of four weeks of content modules on
a wide variety of topics. Each module was screened with a focus on where
hands-on exercises could be added or expanded without substantially altering the
module structure or material. Modules focus primarily on usage of proprietary
software like ArcGIS were ignored. Table \ref{tab:tContentAnalysis} shows the
modules where hands-on potential was identified.

\begin{table}[!htbp]
  \centering
  \caption{Content analysis of existing lectures for hands-on potential.}
  \label{tab:tContentAnalysis}
  \begin{tabularx}{\textwidth}{lX}
    \toprule
    % header row start
    \textbf{Module} & \textbf{Hands-on potential} \\
    \midrule
    % row start
    Interoperabilität &
      \begin{itemize}[left=0pt,nosep,before={\begin{minipage}[t]{\hsize}},after
      ={\end{minipage}}]
      \item Download data from WFS and save as Shapefile.
      \item Inspection and conversion of formats using gdalinfo and ogr2ogr.
      \item Read Shapefile into PostGIS. \end{itemize}\nointerlineskip\\
    \midrule
    % row start
    SQL & Write and run basic SQL queries on sample data. \\
    \midrule
    % row start
    Geodatenbanken &
    \begin{itemize}[left=0pt,nosep,before={\begin{minipage}[t]{\hsize}},after
    ={\end{minipage}}]
      \item Run basic spatial queries on sample data.
      \item Perform spatial aggregation on sample data.
      \item Optional: Connect spatial database to QGIS.
      \end{itemize}\nointerlineskip\\
    \midrule
    % multirow(4) start
    Geometrische Methoden & \multirow[t]{4}{*}{Run simple examples on sample
    data.} \\
    \cmidrule(r){1-1} Topologische Methoden &  \\
    \cmidrule(r){1-1} Mengenmethoden &  \\
    \cmidrule(r){1-1} Statistische Methoden &  \\
    \midrule
    % row start
    Einführung in Python & Self-study introduction via Jupyter Notebook. \\
    \midrule
    % row start
    Internet und GIS I \& II &
      \begin{itemize}[left=0pt,nosep,before={\begin{minipage}[t]{\hsize}},after
      ={\end{minipage}}]
      \item Load data from WMS/WFS/WCS.
      \item Optional: Consume WMS in QGIS. \end{itemize}\nointerlineskip \\
    \midrule
    % row start
    Projekt Internet \& GIS &
      \begin{itemize}[left=0pt,nosep,before={\begin{minipage}[t]{\hsize}},after
      ={\end{minipage}}]
      \item Use geodatabase as data source for OGC server.
      \item Create style and serve data as WMS.
      \item Consume WMS with OGC client. \end{itemize}\nointerlineskip \\
    \bottomrule
  \end{tabularx}%
\end{table}%

% Functional requirements
%-------------
\subsection{Functional requirements}
The following functional requirements were identified for the sandbox to be
useful in realizing the hands-on potential identified in the content analysis:
\begin{itemize}
  \item A spatial database is available together with utilities to import and
  query data.
  \item An OGC server implementation is available. There is a way to read
  predefined example data, read data from the user's system or connect to the
  sandbox spatial database.
  \item An OGC client implementation is available, which can interact with the
  sandbox OGC server or any other OGC service publicly available on the
  internet.
  \item The GDAL command line utilities, particularly gdalinfo and ogr2ogr, are
  available. There is a way to read either predefined example data or read data
  from the user's system.
  \item An environment to write and run python scripts which make use of the
  python geo-ecosystem. A use case would be to provide illustrative examples for
  the concepts of geometry, topology and spatial statistics.
\end{itemize}

% Non-Functional requirements
%-------------
\subsection{Non-Functional requirements}
\begin{itemize}
  \item Users can create content which is persisted between usages of the
  sandbox.
  \item There is a possibility to ship example scripts and data as part of the
  sandbox.
  \item Creation and distribution of data stories that are compatible with the
  tools of the sandbox should be possible without in-depth technological
  knowledge.
  \item The sandbox can be installed effortlessly and in a uniform way across
  the most common operating systems, namely Apple OS, Linux and Windows.
  \item The sandbox can be uninstalled without leaving traces on the operating
  system.
  \item Turning the sandbox on/off and accessing components should not require
  any programming knowledge.
  \item A partial or total reset of applications and data is possible in case
  something got messed up.
  \item Usage is not coupled to the duration of the CAS course or otherwise
  restricted (e.g. via externally controlled credentials).
\end{itemize}

%-------------
% SANDBOX
%-------------
\section{Sandbox}

% Existing projects
%-------------
\subsection{Existing projects}
The Google search engine was used to look for already existing similar projects
using various combinations of the terms \emph{gis}, \emph{geo}, \emph{sandbox},
\emph{workshop}, \emph{playground} and \emph{spatial}. No project was found that
satisfied the requirements outlined in section \ref{sectionrequirements}, but
some noteworthy findings are listed below.

\begin{itemize}
  \item \href{https://joeyklee.github.io/geosandbox/}{Joeyklee's Geosandbox} is
  a collection of tutorials which seems to focus on JavaScript client side
  mapping libraries such as Leaflet. It is not suitable for our purpose but
  mentioned here for the very similar name Geosandbox to avoid confusion.
  \item \href{https://sandbox.idre.ucla.edu/sandbox/category/projects}{UCLA
  sandbox} does not provide any easy access to tools, but seems to be a
  collection of workshops and tutorials. It is mentioned here because the
  interesting goals and presentation of some workshops served as inspiration for
  the data story of this thesis.
  \item \href{https://www.dea.ga.gov.au/developers/sandbox}{Digital Earth
  Australia (DEA) Sandbox} was a wonderful discovery. After creating a free
  account a cloud hosted JupyterJab platform with extensive python notebooks
  provide an interactive engaging experience to learn about the analysis of
  satellite imagery and much more. The DEA sandbox was a major inspiration for
  the heavy focus use of JupyterLab in our sandbox.
  \item \href{https://github.com/geopython/geopython-workshop}{Geopython
  Workshop Repository} is an extensive knowledge hub about a broad range of
  geospatial topics. The workshop is built on python notebooks using a
  containerized JupyterLab environment with pre-installed libraries. Docker
  compose is used as container orchestration technology. Even though the setup
  is too limited for the use case of this thesis (a spatial database is missing
  for example) the idea of using container technology to abstract away
  dependency management provided major inspiration for this thesis.
\end{itemize}

% Technology screening
%-------------
\subsection{Technology evaluation}
The research on existing projects and further exploration of potential
technologies led to three distinct groups of architectural approaches. The
approaches are a) Sandbox as direct installation on user system, b) Sandbox as
local containerized application and c) Sandbox as cloud service.

\subsubsection*{Sandbox as direct installation on user system}
It was found that only four established open-source projects (QGIS, PostGIS,
pgAdmin and Geoserver) would be enough to satisfy the major part of the
functional requirements. While the roles of PostGIS as spatial database and
Geoserver as OGC data server are well-defined, the QGIS installation is
particularly interesting because it comes bundled with versions of GDAL and
python that work together well and provides appropriate interfaces. The
community has produced installers for all major operating systems, which makes
the installation relatively straight-forward.

When it comes to non-functional requirements this approach falls short. A major
shortcoming is the lengthy and intrusive setup stemming from the lack of
separation of the sandbox software from the rest of the user's system. Depending
on already existing software and specific configurations this approach is at
best error-prone and at worst capable to permanently alter the user's system.
Because with this approach only the tools themselves are installed, tutorials
and data stories would need to be provided in a separate way. On the positive
side, the installed tools could be used beyond the duration of the CAS course
without problems and the decoupling of the distribution of tutorials and data
stories make it easy to be updated without technical knowledge.


\subsubsection*{Sandbox as local containerized application}


\subsubsection*{Sandbox as cloud service}


% Docker compose setup
%-------------
\subsection{Docker compose setup}
\begin{figure}[H]
  \centering
  \includegraphics[width=1\textwidth]{composeSetup}
  \caption{Sandbox setup using docker compose}
  \label{fig:sandboxsetup}
\end{figure}

%-------------
% DATA STORY
%-------------
\section{Data story}

%-------------
% CONCLUSION
%-------------
\section{Conclusion}


\end{document}
