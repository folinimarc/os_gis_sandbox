\documentclass[11pt, a4paper, oneside, parskip=full-]{scrartcl}

% -- PREABLE
\usepackage[utf8]{inputenc}
\usepackage{graphicx}

\graphicspath{ {img/} } % path where grahics are stored

% title page parameters
\title{Facilitate experimentation in educational GIS workshops by using Sandboxes \& Data Stories}
\author{Marc Folini}
\date{May 2022}

% -- BODY
\begin{document}

%-------------
% TITLE, ABSTRACT
%-------------
\begin{titlepage}
  \pagenumbering{Roman}
  \setcounter{page}{1}
  % create title page
  \clearpage\maketitle
  \thispagestyle{empty}
  % abstract
  \begin{abstract}
    It is generally desirable in educational GIS workshops to supplement the theoretical concepts with hands-on
    exercises in order for participants to become an active part in the learning experience. This requires that
    participants have access to an environment that provides the necessary software and data - ideally with minimal
    setup effort, independent of the system they use and easy to uninstall without leaving traces after the workshop. We
    evaluated different approaches and found container technology to be a good fit to abstract away the hassle of
    setting up the tools and infrastructure. We provide an implementation of what we call a \emph{sandbox environment}
    and publish it as an open source project.

    Furthermore, this thesis explores the concept of \emph{data stories}. A data story provides selected raw data, a
    motivating tangible goal and instructions that connect the raw data to the goal. This allows participants to fully
    focus on how the available tools interact with data and other tools to achieve something useful. We implement an
    example data that covers aggregating data in PostGIS and serving it as web service using Geoserver.
  \end{abstract}
\end{titlepage}

%-------------
% TABLE OF CONTENT
%-------------
\newpage
\tableofcontents

%-------------
% INTRODUCTION
%-------------
\newpage
\pagenumbering{arabic}
\setcounter{page}{1}
\section{Introduction}
Educational GIS\footnote{Geographic Information System} workshops face a few particular challenges, two of which will be
central to this thesis. Firstly, it is generally desirable that participants can immediately experience theoretical
concepts hands-on for themselves. The challenge here lies in providing all participants with access to the respective
software and data with minimal setup efforts. Secondly, the GIS landscape today consists of many components covering the
whole data lifecycle from initial data exploration to visualization and distribution. Together with the myriad of
commercial and open source tools for each component, this might be overwhelming to unfamiliar participants.

This thesis is written as part of the CAS RIS 2021/22\footnote{Certificate of advanced studies (CAS) in GIS (in german
RIS) offered by ETH Zurich.}. It explores the two above-mentioned challenges in the context of the CAS and aims to
address them on two different levels:
\begin{enumerate}
  \item \textbf{Infrastructure} In order to provide participants access to a variety of software with minimal setup
  effort the concept of a \emph{sandbox environment} is adopted. The requirements and components to be implemented in a
  first version will be chosen to support the content of this thesis' further education curriculum.
  \item \textbf{Content} In order to help participants to make sense of the sandbox components and their interactions,
  we introduce the concept of a \emph{data journey}. Each data journey has a tangible data use case goal statement that
  participants can identify with, for example: Let's visualize risk of wildfire and serve it as interactive map on a
  website. Starting from given raw data, participants follow the processing of this dataset through the different
  components and experience how they work together to achieve the goal.
\end{enumerate}


%-------------
% REQUIREMENTS
%-------------
\section{Requirements}
In this section we first analyze the content of the first four weeks of the CAS RIS and assess potential for hands-on
exercises. From these findings we then derive functional requirements\footnote{Functional requirements specify the tasks
an application must be able to perform.}.


\subsection{Content analysis}


\subsection{Functional requirements}

\subsection{Non-Functional requirements}


%-------------
% SANDBOX
%-------------
\section{Sandbox}

%-------------
% DATA STORY
%-------------
\section{Data Story}

%-------------
% CONCLUSION
%-------------
\section{Conclusion}


\end{document}